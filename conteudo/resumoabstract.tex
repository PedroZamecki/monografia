%!TeX root=../tese.tex
%("dica" para o editor de texto: este arquivo é parte de um documento maior)
% para saber mais: https://tex.stackexchange.com/q/78101

% As palavras-chave são obrigatórias, em português e em inglês, e devem ser
% definidas antes do resumo/abstract. Acrescente quantas forem necessárias.
\palavraschave{Jogo, Corrida, OpenGL, Kart}

\keywords{Game, Racing, OpenGL, Kart}

% O resumo é obrigatório, em português e inglês. Estes comandos também
% geram automaticamente a referência para o próprio documento, conforme
% as normas sugeridas da USP.
\resumo{
Este trabalho apresenta o desenvolvimento de USP Kart, um jogo eletrônico inspirado no clássico Mario Kart. O projeto combina elementos de entretenimento e desafios técnicos, integrando conceitos de computação gráfica em OpenGL moderno e algoritmos para o controle dos jogadores, em meio a um ambiente extremamente dinâmico de um jogo em tempo real.

Utilizando bibliotecas amplamente empregadas no desenvolvimento de gráficos 3D, como OpenGL, GLEW, GLFW, ASSIMP e GLM, USP Kart foi implementado em C++ para explorar os recursos de paralelismo da linguagem. Este recurso foi essencial para gerenciar múltiplos corredores controlados por computador, representando o algoritmo programado para competir no jogo. O comportamento desses agentes foi desenvolvido utilizando um modelo de representação de carro simplificado, conhecido como modelo de bicicleta 2D (em inglês, \textit{bicycle model}: \cite{bicycleModel}), em combinação com o algoritmo A*, que permite o cálculo eficiente do caminho mais curto até a próxima linha de controle, mesmo em cenários dinâmicos e complexos. Outro recurso implementado corresponde a IA de elástico (em inglês: \textit{rubber band AI}: \cite{rubberBandAi}), que se refere a um recurso de inteligência artificial encontrado em títulos de corrida projetado para evitar que os jogadores fiquem muito à frente dos oponentes controlados pelo computador, de modo a manter um nível consistente de desafio do início ao fim da corrida.

A estética do jogo foi criada do zero, com uma direção artística que homenageia a Universidade de São Paulo (USP) e o Instituto de Matemática e Estatística (IME), incluindo a representação da mascote Fluffy. Essa identidade visual foi projetada para proporcionar uma experiência imersiva e personalizada para os jogadores.

O projeto também aborda os desafios encontrados na integração entre gráficos, física, ferramentas para desenvolvimento e cálculos de descoberta de caminhos em tempo real, além de destacar os benefícios do uso de técnicas de programação modernas e bibliotecas de código aberto. Por fim, USP Kart busca não apenas oferecer um produto jogável, mas também servir como estudo de caso para a aplicação prática de conceitos de computação gráfica, paralelismo e algoritmos de busca.
}
\abstract{
This work presents the development of USP Kart, an electronic game inspired by the classic Mario Kart. The project combines elements of entertainment and technical challenges, integrating concepts of computer graphics in modern OpenGL and algorithms for player control, in the midst of an extremely dynamic real-time game environment.

Using widely employed libraries in the development of 3D graphics, such as OpenGL, GLEW, GLFW, ASSIMP, and GLM, USP Kart was implemented in C++ to explore the language's parallelism features. This feature was essential to manage multiple computer-controlled racers, representing the algorithm programmed to compete in the game. The behavior of these agents was developed using a simplified car representation model, known as the 2D bicycle model (\cite{bicycleModel}), in combination with the A* algorithm, which allows the efficient calculation of the shortest path to the next checkpoint, even in dynamic and complex scenarios. Another implemented feature corresponds to Rubber Band AI (\cite{rubberBandAi}), which refers to an artificial intelligence feature found in racing titles designed to prevent players from getting too far ahead of computer-controlled opponents, in order to maintain a consistent level of challenge from start to finish.

The game's aesthetics were created from scratch, with an artistic direction that pays homage to the University of São Paulo (USP) and the Institute of Mathematics and Statistics (IME), including the representation of the Fluffy mascot. This visual identity was designed to provide an immersive and personalized experience for players.

The project also addresses the challenges encountered in the integration between graphics, physics, development tools, and real-time pathfinding calculations, as well as highlighting the benefits of using modern programming techniques and open-source libraries. Finally, USP Kart seeks not only to offer a playable product but also to serve as a case study for the practical application of computer graphics, parallelism, and search algorithms.
}
