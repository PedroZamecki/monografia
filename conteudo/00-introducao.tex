%!TeX root=../tese.tex
%("dica" para o editor de texto: este arquivo é parte de um documento maior)
% para saber mais: https://tex.stackexchange.com/q/78101

%% ------------------------------------------------------------------------- %%

% "\chapter" cria um capítulo com número e o coloca no sumário; "\chapter*"
% cria um capítulo sem número e não o coloca no sumário. A introdução não
% deve ser numerada, mas deve aparecer no sumário. Por conta disso, este
% modelo define o comando "\chapter**".
\chapter**{Introdução}
\label{cap:introducao}

\enlargethispage{.5\baselineskip}

A indústria de jogos eletrônicos é um dos setores que mais cresce globalmente, movimentando bilhões de dólares e impactando diversas áreas, desde a tecnologia até a cultura popular. Com mais de quarenta anos de história, os jogos eletrônicos evoluíram de experiências simples em 2D para produções altamente complexas e imersivas em 3D, alavancadas por avanços tecnológicos e pela criatividade de desenvolvedores ao redor do mundo. Gêneros como aventura, RPG (\textit{Role-playing game}, ou em português 'jogo de interpretação de papéis'), FPS (\textit{First Person Shooter}, ou em português 'Tiro em primeira pessoa') e corrida se consolidaram como pilares dessa indústria, atraindo uma ampla gama de jogadores.

No gênero de jogos de corrida, títulos como \textit{Mario Kart}\index{Língua estrangeira} se destacam não apenas pela jogabilidade divertida e acessível, mas também pela capacidade de reunir amigos e familiares em competições intensas e, por vezes, imprevisíveis. Lançado originalmente em 1992, \textit{Mario Kart} (\cite{marioKart}) redefiniu o gênero ao introduzir elementos de combate e interação dinâmica, como o uso de itens para atrapalhar os adversários. Esse modelo de jogo se mostrou extremamente popular e influente, servindo de inspiração para outros títulos como \textit{Crash Team Racing}\index{Língua estrangeira} (\cite{crashTeamRacing}), \textit{Sonic \& All-Stars Racing}\index{Língua estrangeira} (\cite{sonicAllStars}) e \textit{Blur}\index{Língua estrangeira} (\cite{blur}). Ademais, estudos recentes mostram que \textit{Mario Kart} é considerado um dos jogos mais estressantes pelos jogadores (\cite{stressful:bonusFinder}), devido às viradas de sorte e à competitividade elevada, aspectos que o tornam cativante e memorável.

\section**{Motivação}
Apesar do sucesso internacional desse estilo de jogo, o mercado brasileiro ainda carece de produções relevantes no segmento. A indústria nacional de jogos tem mostrado crescimento expressivo, mas a maior parte dos títulos lançados se concentra em gêneros como aventura, \textit{indie}\index{Língua estrangeira} e simuladores. Jogos de corrida com o apelo dinâmico de \textit{Mario Kart}, especialmente aqueles com elementos culturais que reflitam a identidade brasileira, são raros ou inexistentes. Esse cenário evidencia uma lacuna que pode ser explorada por desenvolvedores locais, tanto para diversificar o mercado quanto para criar experiências que dialoguem com o público brasileiro.

O presente trabalho apresenta uma primeira contribuição para suprir a carência de títulos nacionais sobre o tema com o desenvolvimento do \textit{USP Kart}, um jogo de corrida inspirado em \textit{Mario Kart}, mas com uma identidade fortemente vinculada à Universidade de São Paulo (USP) e ao Instituto de Matemática e Estatística (IME). Além de oferecer uma experiência de jogo envolvente, o projeto também explora soluções tecnológicas inovadoras e visa destacar o potencial criativo da indústria brasileira de jogos. A partir dessa monografia, espera-se não apenas preencher a lacuna existente no mercado nacional, mas também contribuir para o avanço do conhecimento técnico e artístico no desenvolvimento de jogos eletrônicos no Brasil.

\section**{Objetivos}
\label{sec:consideracoes_preliminares}
O objetivo principal deste trabalho é desenvolver o \textit{USP Kart}, um jogo eletrônico de corrida inspirado no clássico \textit{Mario Kart}, com características únicas que valorizem a identidade cultural e acadêmica da Universidade de São Paulo (USP). Para alcançar essa meta, os seguintes objetivos específicos são estabelecidos:
\begin{itemize}
\item Criar uma experiência de jogo imersiva e divertida, utilizando mecânicas de corrida e combate dinâmicas que remetam a jogos consagrados do gênero.

\item Desenvolver a estética do jogo com temática inspirada na USP, com ênfase especial no Instituto de Matemática e Estatística (IME), incluindo referências visuais e conceituais à mascote Fluffy e outros elementos distintivos.

\item Implementar tecnologias modernas de computação gráfica e inteligência artificial para garantir a qualidade técnica e inovadora do jogo.

\item Explorar o potencial de jogos eletrônicos para destacar a cultura universitária brasileira e preencher lacunas no mercado nacional de jogos eletrônicos.
\end{itemize}

\enlargethispage{.5\baselineskip}

\section**{Metodologia}

O desenvolvimento do \textit{USP Kart} seguiu uma abordagem estruturada que combina aspectos técnicos e criativos. O trabalho foi conduzido em etapas distintas, conforme descrito a seguir:

\begin{itemize}

\item \textbf{Criação de um motor gráfico do zero em C++:} o jogo foi desenvolvido utilizando um motor gráfico proprietário, desenvolvido especificamente para atender às necessidades do projeto. Essa abordagem permitiu maior controle sobre os detalhes técnicos, possibilitando assim a exploração de funcionalidades customizadas.
    
\item \textbf{Utilização do OpenGL:} o OpenGL foi empregado como base para a renderização gráfica, garantindo alto desempenho e flexibilidade no desenvolvimento dos recursos visuais do jogo. A escolha dessa tecnologia permite a criação de gráficos 3D imersivos e de alta qualidade.
    
\item \textbf{Suporte multiplataforma:} o jogo foi projetado para ser compatível com múltiplos sistemas operacionais, incluindo Windows, macOS e Linux. Isso foi alcançado por práticas de programação portáveis e pela utilização de bibliotecas compatíveis com diferentes plataformas.
    
\item \textbf{Integração de elementos estéticos e culturais:} durante o desenvolvimento, foi dado um enfoque especial à estilização do jogo com temas relacionados à USP e ao IME, criando uma identidade visual e narrativa que destacam elementos culturais e acadêmicos.
    
\end{itemize}
    
Essa metodologia visa assegurar que o \textit{USP Kart} seja tecnicamente robusto, esteticamente atrativo e acessível a um público diversificado.
