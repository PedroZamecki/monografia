\chapter{Resultados}

O desenvolvimento do \textit{USP Kart} foi concluído com sucesso, resultando em um jogo de corrida estilo \textit{Kart} com temática da Universidade de São Paulo (USP). Foi desenvolvido em C++ com OpenGL moderno e tem como principal característica a corrida com karts em um ambiente universitário, com personagens inspirados na mascote Fluffy.

O jogo foi desenvolvido em duas máquinas, uma com Windows 11 e outra com Fedora Linux 41 (Workstation), ou seja, tudo o que foi desenvolvido deve ser compatível com ambos os sistemas operacionais.

\section{Desempenho}

Percebe-se que o jogo tem algumas partes que devem ser otimizadas para rodar em computadores mais antigos, já que se utiliza de várias \textit{threads}, não é possível garantir que o jogo rode em computadores com poucos núcleos de processamento.

Os resultados obtidos foram uma utilização muito grande do processador e muito pequena da placa de vídeo, o que indica que o jogo é muito dependente do processador, e que a placa de vídeo não é utilizada eficientemente.

\section{Arte}

A arte do jogo ficou bem simples e não tão uniforme, já que foi elaborada por um desenvolvedor sem experiência em modelagem 3D. A textura dos personagens é simples, com cores sólidas e sem muitos detalhes, mas a escolha artística não foi tão satisfatória quanto o esperado, já que poucos modelos foram feitos concretamente, fazendo com que o jogo pareça inacabado.

\section{Jogabilidade}

O jogo tem excelente jogabilidade, com controles simples e bem responsivos, o que permite que o jogador se concentre na corrida. O sistema de colisão é simples, com certas falhas, mas eficaz, permitindo que o personagem colida com objetos e outros personagens.

O jogador não possui nenhuma ação além de acelerar, frear e virar, o que simplifica a jogabilidade e permite que o jogador se concentre na corrida. Essas mesmas ações são utilizadas para todos os personagens, controlados pelo jogador ou não, fazendo com que as corridas sejam justas e equilibradas.

Infelizmente a inteligência artificial não processa informações rápido o suficiente, fazendo com que o jogo pareça mais fácil do que deveria ser, já que os personagens controlados pelo computador dependem muito de um excelente desempenho no computador atualmente, o que não foi o caso com máquinas mais simples.

O jogo também não contou com habilidades especiais, o que poderia tornar o jogo mais interessante e desafiador, já que o jogador teria que aprender a utilizar essas habilidades especiais para vencer as corridas.

\section{Possíveis melhorias}

Como possíveis melhorias, pode-se citar a otimização do jogo para rodar em computadores mais antigos, a melhoria da arte, mais modelos, a adição de habilidades especiais para os personagens, para tornar o jogo mais interessante e desafiador, e a melhoria da inteligência artificial na totalidade.

\section{Testes e validação}

\textit{USP Kart} foi testado de maneira informal por 10 pessoas diferentes, com idades variadas, e todos os testadores gostaram do jogo, achando-o divertido, mas muito simples, o que indica que o jogo é bom, mas pode ser melhorado.

\section{Conclusão}

O desenvolvimento do \textit{USP Kart} foi concluído com sucesso, resultando em um jogo de corrida estilo \textit{Kart} com temática da Universidade de São Paulo (USP).

Apesar dos problemas encontrados, o jogo é divertido, com controles simples e responsivos. A arte é simples, mas eficaz, e a temática da USP é bem representada.

